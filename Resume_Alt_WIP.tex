% ----------------------------------------------------------------------------------------%
%	Created by Alessandro with TeXShop						%
%	---->	May 27, 2009										%
%	Compiled with XeLaTeX, on Mac OS X						%
%	Licensed under the Creative Commons Attribution 3.0 Unported	%
%	Share, change, spread, and have fun!						%
%	http://creativecommons.org/licenses/by/3.0/					%
%	You can find more at http://aleplasmati.comuv.com				%
% ----------------------------------------------------------------------------------------%

\documentclass[12pt]{article}
%!TEX encoding =  UTF-16
% See geometry.pdf to learn the layout options. There are lots.
\usepackage[hmargin=1.0cm, vmargin=1.0cm]{geometry}               
%symbols - the ones you see on the left of the email and of the phone
\usepackage{bbding}
\usepackage{amssymb}
\usepackage{eurosym}
%Colors/Graphics
\usepackage{color,graphicx}
\usepackage[usenames,dvipsnames]{xcolor}

%Fonts and Tweaks for XeLaTeX
\usepackage{fontspec,xltxtra,xunicode}

\defaultfontfeatures{Mapping=tex-text}
\setromanfont[Mapping=tex-text]{Hoefler Text}
\setsansfont[Scale=MatchLowercase,Mapping=tex-text]{Gill Sans}
%\setmonofont[Scale=MatchLowercase]{Andale Mono}

%Setup hyperref package, and colours for links, text and headings
\usepackage{hyperref}
\definecolor{linkcolour}{HTML}{093DD9}	%light purple link for the email
\definecolor{shade}{HTML}{AEEDF5}	%light blue shade
\definecolor{text1}{HTML}{2b2b2b}		%text is almost black
\definecolor{headings}{HTML}{4D4D4D} 	%dark red

\hypersetup{	colorlinks,breaklinks,
			urlcolor=linkcolour, 
			linkcolor=linkcolour}

\usepackage{fancyhdr}				%custom footer
\pagestyle{fancy}
\fancyhf{}
\cfoot{\color{headings} { \small \today}.}
\renewcommand{\headrulewidth}{0pt}

\usepackage{titlesec}				%custom \section

%CV Sections inspired by: 
%http://stefano.italians.nl/archives/26
\titleformat{\section}
	{\color{headings}
		\scshape\Large\raggedright}{}{0em}{}[\color{black}\titlerule]

\titlespacing{\section}{0pt}{0pt}{5pt}

\begin{document}

\color{text1} % set text color for the whole doc
%<<<< TITLE >>>>

	\par{\centering
		{\sffamily\Huge Ritesh Sinha
	}\\[10pt]	
{\color{headings} 
%		\fontspec[Variant = 2]{Zapfino} 
			 \raggedright\textsc{\emph{Objective:}} To work in a position that best utilises my skills as an IT security professional and provides good opportunities for career growth. 
%				{Vit\fontspec[Variant = 3]{Zapfino}\ae}
			\\[25pt]
			\par}
	{\color{white} \hrule} %does this rule really change anything?
	
\begin{minipage}[t]{0.5\textwidth} %START of left-hand side minipage
\vspace{3pt} %trick for alignment
	
\colorbox{shade}{\textcolor{text1}{
	\begin{tabular}{c|p{8.1cm}}
								& 22, Winton Drive, Glasgow G12 0QA \\
		\raisebox{-3pt}{\Phone}  		&+44.0758.786.9041\\
		\raisebox{-3pt}{\Envelope} 	&\href{mailto:ritesh.kumar.sinha@gmail.com}
								{ritesh.kumar.sinha@gmail.com}
	\end{tabular}
	}
}\\[8pt]
\section{Professional Summary}
\normalsize {\begin{itemize}
  \item 3 years of experience with IT Security assessments
  \item Hands on experience with Web Application and IT Infrastructure security assessments
  \item Have conducted security assessments for a number of financial organisations manually and using open source and commercial tools  
  \item Good understanding of OWASP guidelines and PCI DSS best practices
  \item In depth knowledge of web application related attacks like SQL injection, Cross Site Scripting etc.
\end{itemize}} 

\section{Education}

\begin{tabular}{rl}
 	
	%EDUCATION -1-
	\textsc{Oct} 2011 & Master of Science in\\ 
	  &\textsc{Information Technology}\\
	 & \small\emph{(Expected)}\\
	 & \textbf{University of Glasgow}\\
	 & Glasgow, Scotland \\
	 &\\
 	%EDUCATION -2-
	\textsc{July} 2006 & Bachelor of Engineering in\\
	 & \textsc{Mechanical Engineering}\\
	 & \textbf{Bangalore Institute of Technology},\\
	 & Bangalore, India\\

\end{tabular}\\[10pt]


\section{Skills \& Certifications}
\begin{tabular}{rl}
	\textsc{Languages} :  & C, Java, Python, perl\\
	\textsc{Platforms} :  & Windows, Linux\\
	\textsc{Other tools}: & nmap, Nessus, webscarab,\\
				       &  metasploit, AppScan, \\ 
					& Acunetix WVS, \\
					& HP WebInspect\\
					
	\textsc{Certifications}: & EC Council CEH v5,\\
					  & Honeywell 6sigma \\ & Green Belt Certification.
	\end{tabular}

\end{minipage} %END of left-hand side minipage
\hfill
\begin{minipage}[t]{0.44\textwidth} %START of right-hand side minipage
	\vspace{0pt}	%trick
	
% <<<< NEW SECTION >>>>	
\section{Work Experience}

	%WORK EXPERIENCE -1-
       \raggedleft
	\textsc{\normalsize Oct 2009 -- Sept 2010}\par

	\raggedright\large Honeywell Technology Solutions Lab, Bangalore\\
	\emph{Senior Engineer}\\[5pt]

	\normalsize{Worked in an internal customer facing role supporting Honeywell's security products and internal/external applications from a security perspective. Apart from security testing I was also responsible for security evangelism and driving security from the design phase onwards.}\\[10pt]
%\end{tabular}

	%WORK EXPERIENCE -2-
	\raggedleft
	\textsc{\normalsize Jan 2007 -- Oct 2009}\par

	\raggedright
	\large Microland Private Limited, Bangalore \\
\emph{Senior Engineer}\\[5pt]

\normalsize{Responsibilities included frequent client interaction and providing consultation on various topics related to IT security. Actively involved in all web application security projects. Provided training to new team members. Worked with the Penetration testing team to test and constantly update Penetration testing tool-set and methodologies. Maintained and evaluated all tools based on or related to Linux.}\\[10pt]
\large \emph{Engineer}\\
\normalsize {Designed and implemented a virtual lab from the ground up  with more than a hundred virtual machines for proof of concept demonstrations of vulnerabilities and to simulate client infrastructure. Conducted penetration testing on a number of client networks, created technical reports. Worked with client's technical teams on vulnerability mitigation strategies. Conducted periodic penetration testing of internal web applications.}\\[10pt] 
	
\section{\textsc{Interests}}

	
\normalsize{Open Source Software} \\
	Amateur Photography \\
	Trekking \\	
	
		
\end{minipage} %END of right-hand side minipage

\end{document}  